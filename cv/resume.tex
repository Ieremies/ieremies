%!TEX TS-program = xelatex
%!TEX encoding = UTF-8 Unicode
% Awesome CV LaTeX Template
%
% This template has been downloaded from:
% https://github.com/posquit0/Awesome-CV
%
% Author:
% Claud D. Park <posquit0.bj@gmail.com>
% http://www.posquit0.com
%
% Template license:
% CC BY-SA 4.0 (https://creativecommons.org/licenses/by-sa/4.0/)
%


%%%%%%%%%%%%%%%%%%%%%%%%%%%%%%%%%%%%%%
%     Configuration
%%%%%%%%%%%%%%%%%%%%%%%%%%%%%%%%%%%%%%
%%% Themes: Awesome-CV
\documentclass[]{awesome-cv}
\usepackage{textcomp}
\usepackage{multicol}
%%% Override a directory location for fonts(default: 'fonts/')
\fontdir[fonts/]

%%% Configure a directory location for sections
\newcommand*{\sectiondir}{resume/}

%%% Override color
% Awesome Colors: awesome-emerald, awesome-skyblue, awesome-red, awesome-pink, awesome-orange
%                 awesome-nephritis, awesome-concrete, awesome-darknight
%% Color for highlight
% Define your custom color if you don't like awesome colors
\colorlet{awesome}{awesome-red}
%\definecolor{awesome}{HTML}{CA63A8}
%% Colors for text
%\definecolor{darktext}{HTML}{414141}
%\definecolor{text}{HTML}{414141}
%\definecolor{graytext}{HTML}{414141}
%\definecolor{lighttext}{HTML}{414141}

%%% Override a separator for social informations in header(default: ' | ')
%\headersocialsep[\quad\textbar\quad]
    \begin{document}
    
%%%%%%%%%%%%%%%%%%%%%%%%%%%%%%%%%%%%%%
%     Profile
%%%%%%%%%%%%%%%%%%%%%%%%%%%%%%%%%%%%%%
\begin{center}
	\headerfirstnamestyle{Ieremies} \headerlastnamestyle{Romero} \\
	\vspace{2mm}
	{Professor de Programação e Lógica Computacional} \\
	\vspace{2mm}
	{\faEnvelope\ ieremies@gmail.com} | {\faMobile\ (85) 99214-4733} | {\faMapMarker\ Campinas, SP}
\end{center}
%%%%%%%%%%%%%%%%%%%%%%%%%%%%%%%%%%%%%%
%     Education
%%%%%%%%%%%%%%%%%%%%%%%%%%%%%%%%%%%%%%
/
\cvsection{Formação}
\begin{cventries}
	\cventry
	{Bacharelado em Ciência da Computação}
	{Universidade Estadual de Campinas, UNICAMP}
	{}
	{Fev, 2018 – Atual}
	{\begin{cvitems}
		\item {Aprovado em primeiro lugar no vestibular em 2018.}
		\item {Coeficiente de rendimento no $90\%$ percentil.}
		\item {Antecipação de disciplinas da pós-graduação.}
		\item {Disciplinas eletivas em pedagogia.}
	  \end{cvitems}}
	\cventry
	{Inglês avançado}
	{Instituto Brasil Estados Unidos, IBEU-CE}
	{Fortaleza, CE}
	{Jan, 2014 – Ago, 2016}
	{}
\end{cventries}

\vspace{-2mm}
%%%%%%%%%%%%%%%%%%%%%%%%%%%%%%%%%%%%%%
%     Experience
%%%%%%%%%%%%%%%%%%%%%%%%%%%%%%%%%%%%%%
\cvsection{Experiência}
\begin{cventries}
	\cventry
	{Professor de Programação}
	{Pré-vestibular e Colégio Elite}
	{}
	{Jan, 2022 – Atual}
	{\begin{cvitems}
		\item {Ministrar aulas do itinerário de programação para o ensino médio.}
		\item {Elaboração do currículo do itinerário para todos os anos.}
		\item {Elaboração, correção e acompanhamento das atividades avaliativas.}
		\end{cvitems}}
	\cventry
	{Monitor da graduação}
	{Universidade Estadual de Campinas}
	{}
	{Ago, 2018 – Atual}
	{\begin{cvitems}
		\item {Seis vezes bolsista monitor de disciplinas da graduação (MC102, MC202 e MC358).}
		\item {Tira-dúvidas, monitorias e auxilio na confecção de laboratórios práticos.}
		\end{cvitems}}
	\cventry
	{Estagiário em Engenharia de Software}
	{SF-Labs, Vizio}
	{}
	{Mar, 2022 – Abr, 2022}
	{\begin{cvitems}
		\item {Desenvolvimento em python}
		\item {Metodologias ágeis e arquitetura de sistemas clean.}
		\end{cvitems}}
	\cventry
	{Pesquisador Bolsista}
	{Fundação de Amparo à Pesquisa do Estado de São Paulo}
	{}
	{Ago, 2019 – Jul, 2020}
	{\begin{cvitems}
		\item {Projeto de Iniciação Científica em Teoria dos Jogos Algorítmica.}
		\item {Bolsista com avaliação posterior conceito A.}
		\item {Artigo ``da Silva FJ, Miyazawa FK, Romero IV, Schouery R. Tight Bounds for the Price of Anarchy and Stability in Sequential Transportation Games. arXiv preprint arXiv:2007.08726. 2020 Jul 17.'' submetido.}
		\end{cvitems}}
	\cventry
	{Professor e Coordenador Voluntário}
	{Projeto Além da Escola}
	{}
	{Mar, 2018 – Abr, 2019}
	{\begin{cvitems}
		\item {Aulas de programação para OBI}
		\item {Aulas de matemática para vestibulinho do Cotuca/Etec}
		\item {Coordenador do pré-vestibulinho para escolas técnicas}
		\end{cvitems}}
\end{cventries}

\begin{multicols*}{2}
\cvsection{Projetos}
\begin{cventries}
	\cventry
	{Coordenador do evento no Instituto de Computação.}
	{UNICAMP de Portas Abertas}
	{}
	{}
	{}

	\vspace{-5mm}
	\cventry
	{Ministrado para alunos de Graduação em Computação.}
	{Workshop de Git}
	{}
	{}
	{}

	\vspace{-5mm}
	\cventry
	{Ministrado para alunos de graduação em Engenharia Química.}
	{Workshop de Python}
	{}
	{}
	{\begin{cvitems}
		\item {Uso de python para processamento e análise de dados.}
		\end{cvitems}}

	\vspace{-5mm}
\end{cventries}
\cvsection{Habilidades}
\begin{cventries}
	\cventry
	{}
	{\def\arraystretch{1.15}{\begin{tabular}{ l l }
		Programação:  & {\skill{ Python, C/C++}} \\
		Inglês:  & {\skill{ Conversação, escrita e leitura fluentes}} \\
		\end{tabular}}}
	{}
	{}
	{}
  \end{cventries}
\end{multicols*}
% \cvsection{Premiações}
% \begin{cvhonors}
% 	\cvhonor
% 	{Medalha de prata}
% 	{}
% 	{Olimpíada Brasileira de Informática}
% 	{Jun, 2015}
% \end{cvhonors}
\
\end{document}
